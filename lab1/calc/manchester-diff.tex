Для определения верхней границы частот необходимо найти наиболее высокочастотную составляющую спектра в передаваемом сообщении, которая, в отличие от обычного Манчестерского кода, в дифференциальном Манчестерском коде образуется при передаче как последовательных значений 0, так и чередующихся значений с 1 на 0, при этом период гармонического сигнала (синусоиды), используемого для передачи прямоугольных сигналов 0 и 1, будет равен длительности битового интервала $\tau: T = \tau$, где $\tau$ определяется как величина, обратная значению пропускной способности канала $C: \tau = \frac{1}{C}$. Отсюда верхняя граница частот будет равна \[f_{\text{в}} = \frac{1}{T} = C\]

При применении дифференциального Манчестерского кодирования к наименьшая (нижняя) частота будет образовываться при передаче чередующихся значений с 0 на 1, при этом период гармонического сигнала (синусоиды), используемого для передачи прямоугольных сигналов 0 и 1, будет равен удвоенной длительности битового интервала $\tau: T = 2\tau$. Тогда, нижняя граница частот $f_{\text{н}} = \frac{1}{T} = \frac{C}{2}$.

То есть, при пропускной способности канала связи $C = 10 \, \text{Мбит/с}$ частота основной гармоники равна $f_{\text{в}} = 10 \cdot 10^3 = 10 \, \text{МГц}$, битовый интервал $\tau = 100 \, \text{нс}$, $f_{\text{н}} = 5 \, \text{МГц}$.

Следовательно, спектр: $S = f_{\text{в}} - f_{\text{н}} = 5 \, \text{МГц}$.

Среднее значение частоты передаваемого сообщения находится в интервале $(f_{\text{н}};f_{\text{в}})$ и показывает, какие частоты (низкие или высокие) превалируют в спектре передаваемого сигнала.

Для оценки среднего значения частоты передаваемого сообщения можно для каждого битового интервала определить соответствующую частоту сигнала, просуммировать их и разделить на количество битовых интервалов. В нашем случае: частота основной гармоники $f_0 = C$ соответствует 41 битовому интервалу, частота в 2 раза меньшая, т.е. $\frac{f_0}{2} = \frac{C}{2}$, соответствует одиннадцати битовым интервалам.

Тогда средняя частота рассматриваемого сообщения
\[
	f_{\text{ср}} = \left(41f_0+11\frac{f_0}{2}\right)/ 52 = \frac{93f_0}{104} = \frac{93 \cdot 10}{104} \approx 8.942 \, \text{МГц}
\]

Поскольку середине спектра рассматриваемого сообщения соответствует частота
\[
	f_{1/2} = (f_{\text{н}} + f_{\text{в}}) /2 = \frac{10 + 5}{2} = 7.5 \, \text{МГц}
\]
Можно констатировать, что в спектре сигнала \textit{незначительно превалируют высокие частоты}: $f_{\text{ср}} > f_{1/2}$.

Для качественной передачи двоичных сигналов по реальному каналу связи и возможности их распознавания на приёмной стороне с минимальным количеством ошибок, желательно на передающей стороне формировать сигналы, приближающиеся к прямоугольной форме. Однако, спектр таких сигналов оказывается слишком большим. Можно показать, что для качественного распознавания сигнала на приемной стороне при передаче чередующихся значений 0 и 1 достаточно сформировать сигнал, содержащий первые 4 гармоники (поскольку более высокочастотные гармоники оказывают незначительное влияние на результирующий сигнал) с частотами $f_0=C, f_1=3f_0, f_2=5f_0, f_3=7f_0$. В этом случае верхняя граница частот $f_{\text{в}}=7f_0$, а ширина спектра сигнала при передаче рассматриваемого сообщения соответственно будет равна $S = f_{\text{в}} - f_{\text{н}} = 7f_0-f_0/2=6.5f_0=65 \, \text{МГц}$. Полоса пропускания F, необходимая для качественной передачи данного сообщения, должна быть не менее S, например 65 \, \text{МГц}.

Итак, при пропускной способности канала связи $C = 10 \, \text{Мбит/с}$ верхняя и нижняя границы частот в передаваемом сообщении равны соответственно $f_{\text{в}} = 10 \, \text{МГц}$ и $f_{\text{н}} = 5 \, \text{МГц}$, спектр сигнала $S = 5 \, \text{МГц}$, среднее значение частоты в спектре передаваемого сигнала $f_{\text{ср}} = 8.942 \, \text{МГц}$, полоса пропускания, необходимая для качественной передачи данного сообщения $F=65 \, \text{МГц}$.
